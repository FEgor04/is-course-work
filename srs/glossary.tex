\makenoidxglossaries
\newglossaryentry{task}{
  name={задача},
  description={это минимальная единица работы, которая должна быть выполнена в рамках проекта. В Scrum задачи представляют собой конкретные действия или части работы, требуемые для завершения пользовательской истории или другого крупного элемента, например, тестирования, разработки или исправления ошибки. Каждая задача имеет стату  и может быть назначена одному разработчик.}
}
\newglossaryentry{story}{
  name={стори},
  description={это короткое, простое описание функциональности системы с точки зрения конечного пользователя или клиента. Стори создается для выражения потребности или цели, которые должна удовлетворить команда разработки.}
}
\newglossaryentry{epic}{
  name={эпик},
  description={это крупная пользовательская история или крупный элемент работы, который нельзя выполнить в одном спринте. Эпик обычно разбивается на несколько стори или задач для удобства управления. Он описывает более долгосрочные цели и широкие функции продукта. Эпик может охватывать несколько спринтов или даже несколько версий продукта.}
}
\newglossaryentry{sprint}{
  name={спринт},
  description={это фиксированный отрезок времени (обычно от 1 до 4 недель), в течение которого команда разработки должна выполнить набор задач из бэклога спринта. Спринт начинается с планирования, а заканчивается демонстрацией результата и ретроспективой. Цель спринта — завершить все запланированные задачи и получить инкремент продукта, который можно продемонстрировать стейкхолдерам}
}
\newglossaryentry{product-backlog}{
  name={product backlog},
  description={это упорядоченный и постоянно обновляемый список всего, что планируется сделать для создания и улучшения продукта. Этот артефакт Скрама является единственным источником работы для Скрам-команды. \Gls{product-owner} несет ответственность за Бэклог Продукта, включая его содержимое, доступность и упорядочение}
}
\newglossaryentry{product-owner}{
  name={product Owner},
  description={это одна из 3 зон ответственности в Скрам-команде. Владелец Продукта отвечает за максимизацию ценности продукта, получаемого в результате работы Скрам-команды. В его обязанности также входит курирование и приоритизация Бэклога Продукта. Около 50\% времени Владелец Продукта проводит с клиентами и заинтересованными лицами, остальные 50\% работает совместно с командой}
}
\newglossaryentry{kanban-board}{
  name={kanban доска},
  description={это способ визуализации списка задач. На странице Kanban-доски отображаются все задачи из \Gls{product-backlog} в форме карточек. При этом карточки можно перемещать между статусами.}
}
\newglossaryentry{task-attribute}{
  name={атрибут задачи},
  plural={атрибуты задачи},
  description={это  свойство, ассоциированные с задачей. В данной системе атрибутами задачи являются: название, описание, \gls{task-priority}}
}
\newglossaryentry{task-priority}{
  name={приоритет задачи},
  description={определяет, насколько задача важна для системы. Может быть одним из следующих значений: низкий, средний, высокий, блокер}
}
\newglossaryentry{storypoint}{
  name={сторипоинт},
    description={это условная величина, позволяющая давать Элементам Бэклога относительные веса. Чаще всего для оценки в Стори Поинтах используются числа Фибоначчи (1, 2, 3, 5, 8, 13, …), что позволяет провести оценку достаточно быстро. }
}
