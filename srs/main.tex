\documentclass{article}
\usepackage[T2A]{fontenc}
\usepackage[utf8]{inputenc}
\usepackage{hyperref}
\usepackage[english,russian]{babel}
\author{Егор Федоров \and Андрей Карабанов}
\title{Software Requirements for Zadachnik}
\begin{document}
\maketitle

\section{Описание информационной системы}
Информационная система Zadachnik -- система отслеживания багов и
управления проектами.
Основное предназначение системы -- организовывать управление задачами и багами
в IT проектах, реализующих Scrum.

Информационная система позволит решить следующие задачи:
\begin{itemize}
  \item Отслеживание задач в проекте
  \item Отслеживание связей между задачами
  \item Распределение задач между участникам
  \item Оценка времени на выполнение задач
  \item Оценка времени на выполнение всего проекта
  \item Сбор аналитики по времени выполнения задач
  \item Фасилитация и организация spring planning, daily scrum, spring review и spring retro.
\end{itemize}

\subsection{Классы и характеристики пользователей}
Так как система построена для использования в Scrum-командах, то в ней
можно выделить следующие классы пользователей. Эти классы в основном выделены из
классических ролей в Scrum, в качестве источника был использован веб-сайт scrumtrek.ru
\begin{itemize}
\item Product Owner. Отвечает за максимизацию ценности продукта, получаемого в результате работы Scrum-команды.
В его обязанности также входит курирование и приоритизация бэклога продукта. 
\item Scrum Master. Является лидером-слугой (Servant Leader) для Скрам-команды и для организации в целом.
Обучает команду устранять препятствия, является коучем команды и фасилитирует Мероприятия Скрама.
Фактически является владельцем процесса, ответственным за эффективную работу команды.
\item Developer. Разработчики -- это люди, работающие над Элементами Бэклога Спринта.
Они имеют все необходимые компетенции, чтобы каждый Спринт создавать работающий Инкремент Продукта.
\item Stakeholder. Лицо, дающее обратную связь Владельцу Продукта и Скрам-команде в целом по видению,
Бэклогу Продукта и Инкрементам. Нередко участвует в Обзоре спринта.
\end{itemize}

\end{document}
