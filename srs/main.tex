\documentclass[14pt,a4paper]{extarticle}
\usepackage[T2A]{fontenc}
\usepackage[utf8]{inputenc}
\usepackage{hyperref}
\usepackage{graphicx}
\usepackage[english,russian]{babel}
\usepackage{fancyhdr}

%
\usepackage{tabularx}
\usepackage{array}
\usepackage{longtable}
\usepackage{tabularray}
\usepackage{tikz}

% Inline lists with enumerate*
\usepackage[inline]{enumitem}

% For... glossary
\usepackage[toc,translate=babel]{glossaries}

% To set custom label for enumerate
\usepackage{enumitem}

\usepackage{geometry}
\geometry{right=15mm}
\geometry{left=30mm}
\geometry{bottom=20mm}
\geometry{top=20mm}
\geometry{ignorefoot}% считать от нижней границы текста

\pagestyle{fancy}
\fancyhead{}
\fancyfoot[C]{\thepage}

\renewcommand{\headrulewidth}{0pt}
\fancypagestyle{firststyle}
{
	\fancyhead{}
	\fancyfoot[C]{г. Санкт-Петербург\\2024 г.}
}

\author{Егор Федоров \and Андрей Карабанов}
\title{Software Requirements for Zadachnik}

\usepackage{setspace}
\linespread{1.5}

\setlength{\parindent}{1.25cm} % отступ для абзаца

\makenoidxglossaries
\newglossaryentry{task}{
  name={задача},
  description={это минимальная единица работы, которая должна быть выполнена в рамках проекта. В Scrum задачи представляют собой конкретные действия или части работы, требуемые для завершения пользовательской истории или другого крупного элемента, например, тестирования, разработки или исправления ошибки. Каждая задача имеет стату  и может быть назначена одному разработчик.}
}
\newglossaryentry{story}{
  name={стори},
  description={это короткое, простое описание функциональности системы с точки зрения конечного пользователя или клиента. Стори создается для выражения потребности или цели, которые должна удовлетворить команда разработки.}
}
\newglossaryentry{epic}{
  name={эпик},
  description={это крупная пользовательская история или крупный элемент работы, который нельзя выполнить в одном спринте. Эпик обычно разбивается на несколько стори или задач для удобства управления. Он описывает более долгосрочные цели и широкие функции продукта. Эпик может охватывать несколько спринтов или даже несколько версий продукта.}
}
\newglossaryentry{sprint}{
  name={спринт},
  description={это фиксированный отрезок времени (обычно от 1 до 4 недель), в течение которого команда разработки должна выполнить набор задач из бэклога спринта. Спринт начинается с планирования, а заканчивается демонстрацией результата и ретроспективой. Цель спринта — завершить все запланированные задачи и получить инкремент продукта, который можно продемонстрировать стейкхолдерам}
}
\newglossaryentry{product-backlog}{
  name={product backlog},
  description={это упорядоченный и постоянно обновляемый список всего, что планируется сделать для создания и улучшения продукта. Этот артефакт Скрама является единственным источником работы для Скрам-команды. \Gls{product-owner} несет ответственность за Бэклог Продукта, включая его содержимое, доступность и упорядочение}
}
\newglossaryentry{product-owner}{
  name={product Owner},
  description={это одна из 3 зон ответственности в Скрам-команде. Владелец Продукта отвечает за максимизацию ценности продукта, получаемого в результате работы Скрам-команды. В его обязанности также входит курирование и приоритизация Бэклога Продукта. Около 50\% времени Владелец Продукта проводит с клиентами и заинтересованными лицами, остальные 50\% работает совместно с командой}
}
\newglossaryentry{kanban-board}{
  name={kanban доска},
  description={это способ визуализации списка задач. На странице Kanban-доски отображаются все задачи из \Gls{product-backlog} в форме карточек. При этом карточки можно перемещать между статусами.}
}
\newglossaryentry{task-attribute}{
  name={атрибут задачи},
  plural={атрибуты задачи},
  description={это  свойство, ассоциированные с задачей. В данной системе атрибутами задачи являются: название, описание, \gls{task-priority}, статус, исполнитель и оценка задачи в \glsdisp{storypoint}{сторипоинтах}}
}
\newglossaryentry{task-priority}{
  name={приоритет задачи},
  description={определяет, насколько задача важна для системы. Может быть одним из следующих значений: низкий, средний, высокий, блокер}
}
\newglossaryentry{storypoint}{
  name={сторипоинт},
  description={это условная величина, позволяющая давать Элементам Бэклога относительные веса. Чаще всего для оценки в Стори Поинтах используются числа Фибоначчи (1, 2, 3, 5, 8, 13, …), что позволяет провести оценку достаточно быстро. }
}


\begin{document}
\begin{titlepage}
	\thispagestyle{firststyle}
	\begin{center}
		ФЕДЕРАЛЬНОЕ ГОСУДАРСТВЕННОЕ АВТОНОМНОЕ ОБРАЗОВАТЕЛЬНОЕ УЧРЕЖДЕНИЕ ВЫСШЕГО ОБРАЗОВАНИЯ\\
		\vspace{0.5cm}
		<<Национальный исследовательский университет ИТМО>>\\
		Факультет Программной Инженерии и Компьютерной Техники \\
		\vspace{1cm}
	\end{center}

	\vspace{1cm}

	\begin{center}
		\large
		\textbf{Курсовая работа}\\
		по дисциплине\\
		\textbf{<<Информационные системы>>} \\
	\end{center}

	\vspace{2cm}

	\begin{flushright}
		Выполнили студенты  группы P3315\\
		\textbf{Федоров Егор Владимирович} \\
		\textbf{Карабанов Андрей Федорович} \\
		Преподаватель: \\
		\textbf{Егошин Алексей Васильевич}\\
	\end{flushright}
\end{titlepage}

\tableofcontents

\section{Описание предметной области}
В системе Scrum люди делятся по ролям
Developer, Scrum Master, Product Owner.
Для организации работы создаются \glsdisp{epic}{эпики}.
Эпик может содержать внутри себя \glsdisp{story}{стори} и
\glsdisp{task}{задачи}. Стори также может содержать в себе задачи.
Эпики, стори и задачи имеют \glsdisp{task-attribute}{атрибуты}.
Задачи также могут иметь сложные связи между собой, например
задача может блокировать другую или задача может являться клоном другой.

Также в Scrum имеются 4 церемонии-встречи: daily meeting, sprint
planning, spring review, sprint retro.
Встречи также могут иметь минутки -- записи о всем, что произошло на встрече.
Для оценки времени на выполнение задач в Scrum используются \glsdisp{storypoint}{сторипоинты}.

Люди объединяются в команды. Команда обязательно имеет одну и только одну доску.
Команда имеет особый идентификатор.
Список задач команды в основном управляется scrum master.

При этом команда может участвовать в нескольких продуктах.
Задачи создаются в продуктах и тем самым формируют его бэклог.
Список задач продукта в основном управляется product owner.

Для выполнения задач команды создают спринты и на каждый
спринт формируют бэклог спринта.
В процессе спринта команда \emph{может} создать релиз.
По окончанию спринта команда \emph{обязана} создать релиз.

Релиз -- описание всех изменений в продукте.
Описание изменений состоит из списка всех задач,
закрытых при релизе и типов этих изменений (добавлено, изменено и т.п., см. сайт \url{keepachangelog.org}).
Релиз также может содержать release notes - дополнительное описание релиза, например
краткое описание всех изменений.

\section{Описание информационной системы}
Информационная система Zadachnik -- система отслеживания багов и
управления проектами.
Основное предназначение системы -- организовывать управление задачами и багами
в IT проектах, реализующих Scrum.

Информационная система позволит решить следующие задачи:
\begin{itemize}
	\item Отслеживание задач в проекте
	\item Отслеживание связей между задачами
	\item Распределение задач между участникам
	\item Оценка времени на выполнение задач
	\item Оценка времени на выполнение всего проекта
	\item Сбор аналитики по времени выполнения задач
	\item Фасилитация и организация sprint planning, daily scrum, spring review и spring retro.
\end{itemize}

\subsection{Классы и характеристики пользователей}
Так как система построена для использования в Scrum-командах, то в ней
можно выделить следующие классы пользователей. Эти классы в основном выделены из
классических ролей в Scrum, в качестве источника был использован веб-сайт scrumtrek.ru
\begin{itemize}
	\item Product Owner. Отвечает за максимизацию ценности продукта, получаемого в результате работы Scrum-команды.
	      В его обязанности также входит курирование и приоритизация бэклога продукта.
	\item Scrum Master. Является лидером-слугой (Servant Leader) для Скрам-команды и для организации в целом.
	      Обучает команду устранять препятствия, является коучем команды и фасилитирует Мероприятия Скрама.
	      Фактически является владельцем процесса, ответственным за эффективную работу команды.
	\item Developer. Разработчики -- это люди, работающие над Элементами Бэклога Спринта.
	      Они имеют все необходимые компетенции, чтобы каждый Спринт создавать работающий Инкремент Продукта.
\end{itemize}

\section{Функциональные требования}

В данной секции в формате user story описаны требования к системе.
Требования разбиты на подсекции в зависимости от класса пользователей,
от которых исходит это требование.
Если требование актуально для нескольких классов пользователей, оно
вынесено в общее.

\subsection{Общие требования}
\begin{enumerate}[label=\textbf{FR\arabic*}.]
	\item Я, как пользователь, хочу иметь возможность зарегистрироваться в системе с помощью логина и пароля
	\item Я, как пользователь, хочу указывать \glsdisp{task-attribute}{атрибуты} задачей, сторей и эпиков
	\item Я, как пользователь, хочу создавать и обновлять задачи, чтобы управлять своими рабочими элементами
	\item Я, как пользователь, хочу менять статус задачи
	\item Я, как пользователь, хочу писать комментарии к минуткам, задачам и релиз-ноутам с помощью редактора текста
	\item Я, как пользователь, хочу просматривать изменения полей задачи
\end{enumerate}

\subsection{Требования Product Owner}
\begin{enumerate}[label=\textbf{POR\arabic*}.]
	\item Я, как Product Owner, хочу создавать релиз продукта
	\item Я, как Product Owner, хочу создавать и приоритизировать бэклог продукта
	\item Я, как Product Owner, хочу настраивать какие команды участвуют в разработке продукта.
	\item Я, как Product Owner, хочу просматривать бэклог продукта в виде Kanban-доски.
	\item Я, как Product Owner, хочу писать release notes к релизу с помощью редактора текста
\end{enumerate}

\subsection{Требования Scrum Master}
\begin{enumerate}[label=\textbf{SMR\arabic*}.]
	\item Я, как Scrum Master, хочу организовывать задачи в спринты
	\item Я, как Scrum Master, хочу начинать спринты
	\item Я, как Scrum Master, хочу заканчивать спринт с выпуском релиза
	\item Я, как Scrum Master, хочу запланировать в системе
	      Sprint Planning Meeting, Scrum Daily Meeting, Sprint Review, Sprint Retrospective
	\item Я, как Scrum Master, хочу настраивать какие разработчики являются частью команды
	\item Я, как Scrum Master, хочу назначать задачи разработчикам.
	\item Я, как Scrum Master, хочу просматривать бэклог спринта в виде Kanban-доски.
	\item Я, как Scrum Master, хочу писать минутки к встречам с помощью редактора текста
	\item Я, как Scrum Master, хочу настраивать Workflow команды.
\end{enumerate}

\subsection{Требования к Kanban-доске}
\begin{enumerate}[label=\textbf{KBR\arabic*}.]
	\item Kanban-доска должна отображать задачи с группировкой по статусу
	\item Kanban-доска должна отображать задачи с сортировкой по приоритету
	\item Kanban-доска должна поддерживать фильтрацию по исполнителю с выбором
	      многих исполнителей
	\item Kanban-доска должна поддерживать фильтрацию по продукту с выбором
	      многих продуктов (checkbox-выбор)
	\item Kanban-доска должна поддерживать фильтрацию по команде с выбором
	      многих команд (checkbox-выбор)
	\item Kanban-доска должна предоставлять возможность изменять статус
	      команд с помощью drag-n-drop между колонками статусов
\end{enumerate}

\subsection{Требования к редактору текста}
\begin{enumerate}[label=\textbf{AER\arabic*}.]
	\item Редактор текста должен предоставлять возможность выделять текст
	      \textbf{жирным} шрифтом, \textit{курсивом} и \texttt{моноширинным} шрифтом
	\item Редактор текста должен поддерживать вставку ссылок на веб-сайты
	\item Редактор текста должен поддерживать отображение заголовков 6-ти разных
	      уровней
\end{enumerate}

\subsection{Требования к Workflow}
\begin{enumerate}[label=\textbf{WFR\arabic*}.]
	\item Система должна предоставлять следующие группы статусов:
	      \begin{enumerate*}
		      \item <<Backlog>>: Backlog - задача не находится ни в одном из бэклогов спринта. To Do - задча находится в бэклоге спринта
		      \item <<Started>>: In Progress - задача взята в работу,
		            In Review - задача находится в ревью,
		            Ready To Merge - задача прошла ревью и готова к мержу
		      \item <<Completed>>: Done - задача выполнена и замержена
		      \item <<Canceled>>: Canceled - задача отменена,
		            Couldn't reproduce - не удалось воспроизвести задачу,
		            Won't Fix - задача не будет исправлена,
		            Duplicate - задача является дубликатом
	      \end{enumerate*}
	\item Я, как Scrum Master, хочу выбирать какие из вышеперечисленных статусов могут быть в дальнейшем назначены задачам
	\item Я, как Scrum Master, хочу добавлять статусы в каждую из вышеперечисленных групп.
	\item Система должна ограничивать максимальное количество активных статусов до 7-ми
\end{enumerate}


\section{Нефункциональные требования}
\begin{enumerate}[label=\textbf{NFR\arabic*}.]
	\item Система должна поддерживать отображения без нарушения работы функциональности
	      и дизайна в браузерах Chrome 130+, Mozilla 120+
\end{enumerate}

\section{Прецеденты использования}
\subsection{Регистрация в сети}
\begin{tabular}{|l|p{9cm}|}
	\hline
	\textbf{Прецендент}            & Регистрация в системе                                                                             \\
	\hline
	\textbf{ID}                    & 1                                                                                                 \\
	\hline
	\textbf{Краткое описание}      & Пользователь регистрируется в cистеме                                                             \\
	\hline
	\textbf{Главный актер}         & Пользователь                                                                                      \\
	\hline
	\textbf{Второстепенные актеры} & нет                                                                                               \\
	\hline
	\textbf{Предусловия}           & Пользователь не имеет учетной записи в системе                                                    \\
	\hline
	\textbf{Основной поток}        & \begin{enumerate}
		                                 \item Пользователь переходит на страницу регистрации
		                                 \item Пользователь вводит свой логин.
		                                 \item Пользователь вводит пароль
		                                 \item Пользователь попадает на страницу настройки своего профиля для продолжения заполнения данных
	                                 \end{enumerate} \\
	\hline
\end{tabular}

\subsection{Вход в систему}

\begin{tabular}{|l|p{9cm}|}
	\hline
	\textbf{Прецендент}            & Вход в систему                                                                                  \\
	\hline
	\textbf{ID}                    & 2                                                                                               \\
	\hline
	\textbf{Краткое описание}      & Пользователь входит в систему, используя свои учетные данные                                    \\
	\hline
	\textbf{Главный актер}         & Пользователь                                                                                    \\
	\hline
	\textbf{Второстепенные актеры} & нет                                                                                             \\
	\hline
	\textbf{Предусловия}           & Пользователь имеет учетную запись в системе                                                     \\
	\hline
	\textbf{Основной поток}        & \begin{enumerate}
		                                 \item Пользователь переходит на страницу входа в систему
		                                 \item Пользователь вводит свой логин.
		                                 \item Пользователь вводит пароль.
		                                 \item Система проверяет введенные данные на корректность
		                                 \item Пользователь попадает на главную страницу системы
	                                 \end{enumerate}                                         \\
	\hline
	\textbf{Альтернативный поток}  & Введенные данные некорректы, система информирует пользователя и предлагает ввести данные заново \\
	\hline
\end{tabular}

\subsection{Создание команды}

\begin{tabular}{|l|p{9cm}|}
	\hline
	\textbf{Прецендент}            & Создание команды                                                         \\
	\hline
	\textbf{ID}                    & 3                                                                        \\
	\hline
	\textbf{Краткое описание}      & Пользователь создает новую комадну                                       \\
	\hline
	\textbf{Главный актер}         & Пользователь с ролью Scrum Master                                        \\
	\hline
	\textbf{Второстепенные актеры} & нет                                                                      \\
	\hline
	\textbf{Предусловия}           & Пользователь авторизован в системе, пользователь имеет роль Scrum Master \\
	\hline
	\textbf{Основной поток}        & \begin{enumerate}
		                                 \item Пользователь переходит на страницу команд
		                                 \item Пользователь нажимает кнопку <<Создать команду>>
		                                 \item Пользователь вводит название команды и идентификатор
		                                 \item Система проверяет введенные данные на корректность
		                                 \item Пользователь попадает на страницу созданной команды
	                                 \end{enumerate}                \\
	\hline
\end{tabular}

\subsection{Создание задачи}

\begin{tabular}{|l|p{9cm}|}
	\hline
	\textbf{Прецендент}            & Создание задачи в бэклоге продукта                                 \\
	\hline
	\textbf{ID}                    & 4                                                                  \\
	\hline
	\textbf{Краткое описание}      & Пользователь создает новую задачу в продукте                       \\
	\hline
	\textbf{Главный актер}         & Пользователь                                                       \\
	\hline
	\textbf{Второстепенные актеры} & нет                                                                \\
	\hline
	\textbf{Предусловия}           & Пользователь авторизован в системе и находится на странице команды \\
	\hline
	\textbf{Основной поток}        & \begin{enumerate}
		                                 \item Пользователь переходит на страницу создание задачи
		                                 \item Пользователь вводит данные для создания задачи
		                                 \item Система проверяет введенные данные на корректность
		                                 \item Пользователь возвращается на страницу доски
	                                 \end{enumerate}            \\
	\hline
\end{tabular}

\subsection{Изменение атрибутов задачи}

\begin{tabular}{|l|p{9cm}|}
	\hline
	\textbf{Прецендент}            & Изменение атрибутов задачи                                         \\
	\hline
	\textbf{ID}                    & 5                                                                  \\
	\hline
	\textbf{Краткое описание}      & Пользователь указывает атрибуты задачи                             \\
	\hline
	\textbf{Главный актер}         & Пользователь                                                       \\
	\hline
	\textbf{Второстепенные актеры} & Зарегистрированные пользователи в системе                          \\
	\hline
	\textbf{Предусловия}           & Пользователь авторизован и имеет доступ к доске задач и задаче     \\
	\hline
	\textbf{Основной поток}        & \begin{enumerate}
		                                 \item Пользователь переходит на страницу доски
		                                 \item Пользователь переходит на страницу задачи
		                                 \item Пользователь выбирает данные для редактирования
		                                 \item Cистема проверяет возможность редактирования выбранных данных
		                                 \item Пользователь вводит новые данные
		                                 \item Система проверяет корректность произведенных действий
	                                 \end{enumerate} \\
	\hline
\end{tabular}

\subsection{Перевод задачи в другой статус}

\begin{tabular}{|l|p{9cm}|}
	\hline
	\textbf{Прецендент}            & Перевод задачи в другой статус                                              \\
	\hline
	\textbf{ID}                    & 6                                                                           \\
	\hline
	\textbf{Краткое описание}      & Пользователь переводит задачу в другой статус на доске задач                \\
	\hline
	\textbf{Главный актер}         & Пользователь                                                                \\
	\hline
	\textbf{Второстепенные актеры} & Зарегистрированные пользователи в системе                                   \\
	\hline
	\textbf{Предусловия}           & Пользователь авторизован и имеет доступ к доске задач                       \\
	\hline
	\textbf{Основной поток}        & \begin{enumerate}
		                                 \item Пользователь переходит на страницу доски
		                                 \item Пользователь перемещает карточку задач в другую колонку на доске задач
		                                 \item Система проверяет корректность произведенных действий
	                                 \end{enumerate} \\
	\hline
\end{tabular}

\subsection{Добавление комментария к задаче}

\begin{tabular}{|l|p{9cm}|}
	\hline
	\textbf{Прецендент}            & Добавление комментария к задаче                            \\
	\hline
	\textbf{ID}                    & 7                                                          \\
	\hline
	\textbf{Краткое описание}      & Пользователь добавляет комментарий к задаче                \\
	\hline
	\textbf{Главный актер}         & Пользователь                                               \\
	\hline
	\textbf{Второстепенные актеры} & Зарегистрированные пользователи в системе                  \\
	\hline
	\textbf{Предусловия}           & Пользователь авторизован и имеет доступ к доске задач      \\
	\hline
	\textbf{Основной поток}        & \begin{enumerate}
		                                 \item Пользователь переходит на страницу доски
		                                 \item Пользователь переходит на страницу задачи
		                                 \item Пользователь переходит в раздел комментариев
		                                 \item Пользователь переходит пишет комментарий
		                                 \item Пользователь сохраняет комментарий
		                                 \item Система проверяет корректность произведенных действий
	                                 \end{enumerate} \\
	\hline
\end{tabular}

\subsection{Добавление комментария к релиз-ноутам}

\begin{tabular}{|l|p{9cm}|}
	\hline
	\textbf{Прецендент}            & Добавление комментария к релиз-ноуту                       \\
	\hline
	\textbf{ID}                    & 8                                                          \\
	\hline
	\textbf{Краткое описание}      & Пользователь добавляет комментарий к релиз-ноуту           \\
	\hline
	\textbf{Главный актер}         & Пользователь                                               \\
	\hline
	\textbf{Второстепенные актеры} & Зарегистрированные пользователи в системе                  \\
	\hline
	\textbf{Предусловия}           & Пользователь авторизован и имеет доступ к релиз-ноуту      \\
	\hline
	\textbf{Основной поток}        & \begin{enumerate}
		                                 \item Пользователь переходит на страницу продукта
		                                 \item Пользователь переходит на страницу релиз-ноутов
		                                 \item Пользователь переходит на страницу релиз-ноута
		                                 \item Пользователь переходит в раздел комментариев
		                                 \item Пользователь переходит пишет комментарий
		                                 \item Пользователь сохраняет комментарий
		                                 \item Система проверяет корректность произведенных действий
	                                 \end{enumerate} \\
	\hline
\end{tabular}

\subsection{Добавление комментария к статье}

\begin{tabular}{|l|p{9cm}|}
	\hline
	\textbf{Прецендент}            & Добавление комментария к статье                            \\
	\hline
	\textbf{ID}                    & 9                                                          \\
	\hline
	\textbf{Краткое описание}      & Пользователь добавляет комментарий к статье                \\
	\hline
	\textbf{Главный актер}         & Пользователь                                               \\
	\hline
	\textbf{Второстепенные актеры} & Зарегистрированные пользователи в системе                  \\
	\hline
	\textbf{Предусловия}           & Пользователь авторизован и имеет доступ к статье           \\
	\hline
	\textbf{Основной поток}        & \begin{enumerate}
		                                 \item Пользователь переходит на страницу статьи
		                                 \item Пользователь переходит в раздел комментариев
		                                 \item Пользователь переходит пишет комментарий
		                                 \item Пользователь сохраняет комментарий
		                                 \item Система проверяет корректность произведенных действий
	                                 \end{enumerate} \\
	\hline
\end{tabular}

\subsection{Просмотр изменения полей задачи}

\begin{tabular}{|l|p{9cm}|}
	\hline
	\textbf{Прецендент}            & Просмотр изменений полей задачи                                \\
	\hline
	\textbf{ID}                    & 10                                                             \\
	\hline
	\textbf{Краткое описание}      & Пользователь просматривает изменения полей задачи              \\
	\hline
	\textbf{Главный актер}         & Пользователь                                                   \\
	\hline
	\textbf{Второстепенные актеры} & Зарегистрированные пользователи в системе                      \\
	\hline
	\textbf{Предусловия}           & Пользователь авторизован и имеет доступ к доске задач и задаче \\
	\hline
	\textbf{Основной поток}        & \begin{enumerate}
		                                 \item Пользователь переходит на страницу доски
		                                 \item Пользователь переходит на страницу задачи
		                                 \item Пользователь переходит в раздел истории изменений
	                                 \end{enumerate}         \\
	\hline
\end{tabular}

\subsection{Создание релиза продукта}

\begin{tabular}{|l|p{9cm}|}
	\hline
	\textbf{Прецендент}            & Создание релиза продукта                                                   \\
	\hline
	\textbf{ID}                    & 11                                                                         \\
	\hline
	\textbf{Краткое описание}      & Product Owner создает релиз продукта                                       \\
	\hline
	\textbf{Главный актер}         & Пользователь с ролью Product Owner                                         \\
	\hline
	\textbf{Второстепенные актеры} & Зарегистрированные пользователи в системе                                  \\
	\hline
	\textbf{Предусловия}           & Пользователь авторизован и имеет роль Product Owner для заданного продукта \\
	\hline
	\textbf{Основной поток}        & \begin{enumerate}
		                                 \item Пользователь переходит на страницу продукта
		                                 \item Пользователь переходит на страницу релизов
		                                 \item Пользователь переходит на страницу создания релиза продукта
		                                 \item Пользователь вводит заметки к релизу, указывает типы задач в релизе
		                                 \item Система проверяет корректность введенных данных
		                                 \item Пользователь сохраняет данные
		                                 \item Система проверяет корректность произведенных действий
		                                 \item Пользователь переходит на страницу релиза
	                                 \end{enumerate}   \\
	\hline
\end{tabular}

\subsection{Создание спринта команды}

\begin{tabular}{|l|p{9cm}|}
	\hline
	\textbf{Прецендент}            & Организация задач в спринты                                                \\
	\hline
	\textbf{ID}                    & 12                                                                         \\
	\hline
	\textbf{Краткое описание}      & Пользователь с ролью Scrum Master создает спринт                           \\
	\hline
	\textbf{Главный актер}         & Пользователь с ролью Scrum Master                                          \\
	\hline
	\textbf{Второстепенные актеры} & Зарегистрированные пользователи в системе                                  \\
	\hline
	\textbf{Предусловия}           & Пользователь авторизован и имеет роль Scrum Master                         \\
	\hline
	\textbf{Основной поток}        & \begin{enumerate}
		                                 \item Пользователь переходит на страницу команды
		                                 \item Пользователь переходит на страницу создания спринта
		                                 \item Пользователь переходит выбирает необходимые задачи
		                                 \item Пользователь указывает даты Scrum Planning, Scrum Review, Scrum Retro
		                                 \item Пользователь указывает продолжительность спринтта
		                                 \item Пользователь сохраняет данные
		                                 \item Система проверяет корректность произведенных действий
		                                 \item Пользователь переходит на доску задач спринта
	                                 \end{enumerate} \\
	\hline
\end{tabular}

\subsection{Приглашение разработчика в команду}

\begin{tabular}{|l|p{9cm}|}
	\hline
	\textbf{Прецендент}            & Назначение разработчиков в команду                                  \\
	\hline
	\textbf{ID}                    & 13                                                                  \\
	\hline
	\textbf{Краткое описание}      & Пользователь с ролью Scrum Master приглашает разработчика в команду \\
	\hline
	\textbf{Главный актер}         & Пользователь с ролью Scrum Master                                   \\
	\hline
	\textbf{Второстепенные актеры} & Зарегистрированные пользователи в системе                           \\
	\hline
	\textbf{Предусловия}           & Пользователь авторизован и имеет роль Scrum Master                  \\
	\hline
	\textbf{Основной поток}        & \begin{enumerate}
		                                 \item Пользователь переходит на страницу команд
		                                 \item Пользователь переходит на страницу команды
		                                 \item Пользователь переходит к настройкам команды
		                                 \item Пользователь нажимает на кнопку "Добавить разработчика"
		                                 \item Пользователь вводит username разработчика
		                                 \item Пользователь сохраняет данные
		                                 \item Система проверяет корректность произведенных действий
		                                 \item Система отправляет приглашение пользователю
	                                 \end{enumerate}        \\
	\hline
\end{tabular}

\section{Даталогическая и инфологическая модели}

\begin{figure}[ht]
  \centering
  \includegraphics[width=0.9\textwidth]{../docs/datalogical.png}
  \caption{Инфологическая модель}
\end{figure}


\clearpage
\begin{figure}[ht]
  \centering
  \includegraphics[width=0.9\textwidth]{../docs/datalogical.png}
  \caption{Даталогическая модель}
\end{figure}

\printnoidxglossaries
\end{document}
