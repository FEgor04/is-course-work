\section{Функциональные требования}

В данной секции в формате user story описаны требования к системе.
Требования разбиты на подсекции в зависимости от класса пользователей,
от которых исходит это требование.
Если требование актуально для нескольких классов пользователей, оно
вынесено в общее.

\subsection{Общие требования}
\begin{enumerate}[label=\textbf{FR\arabic*}.]
  \item Я, как пользователь, хочу иметь возможность зарегистрироваться в системе с помощью логина и пароля
  \item Я, как пользователь, хочу указывать \glsdisp{task-attribute}{атрибуты} задач
  \item Я, как пользователь, хочу создавать и обновлять задачи
  \item Я, как пользователь, хочу менять статус задачи
  \item Я, как пользователь, хочу писать комментарии к минуткам, задачам и релиз-ноутам с помощью редактора текста
  \item Я, как пользователь, хочу просматривать изменения полей задачи
  \item Я, как пользователь, хочу принимать приглашние в команду
\end{enumerate}

\subsection{Требования Product Owner}
\begin{enumerate}[label=\textbf{POR\arabic*}.]
  \item Я, как Product Owner, хочу создавать релиз продукта
  \item Я, как Product Owner, хочу указывать какие задачи вошли в релиз и
        их тип согласно \url{keepachangelog.com}
  \item Я, как Product Owner, хочу писать release notes к релизу с помощью редактора текста
  \item Я, как Product Owner, хочу создавать и приоритизировать бэклог продукта
  \item Я, как Product Owner, хочу приглашать команды в продукт
  \item Я, как Product Owner, хочу просматривать бэклог продукта в виде Kanban-доски.
\end{enumerate}

\subsection{Требования Scrum Master}
\begin{enumerate}[label=\textbf{SMR\arabic*}.]
  \item Я, как Scrum Master, хочу организовывать задачи в спринты
  \item Я, как Scrum Master, хочу начинать спринты
  \item Я, как Scrum Master, хочу заканчивать спринт с выпуском релиза
  \item Я, как Scrum Master, хочу запланировать в системе
        Sprint Planning Meeting, Scrum Daily Meeting, Sprint Review, Sprint Retrospective
  \item Я, как Scrum Master, хочу принимать приглашние в продукт
  \item Я, как Scrum Master, хочу приглашать разработчиков в команду
  \item Я, как Scrum Master, хочу назначать задачи разработчикам.
  \item Я, как Scrum Master, хочу просматривать бэклог спринта в виде Kanban-доски.
  \item Я, как Scrum Master, хочу писать минутки к встречам с помощью редактора текста
  \item Я, как Scrum Master, хочу настраивать Workflow команды.
\end{enumerate}

\subsection{Требования к Kanban-доске}
\begin{enumerate}[label=\textbf{KBR\arabic*}.]
  \item Kanban-доска должна отображать задачи с группировкой по статусу
  \item Kanban-доска должна отображать задачи с сортировкой по приоритету
  \item Kanban-доска должна поддерживать фильтрацию по исполнителю с выбором
        многих исполнителей
  \item Kanban-доска должна поддерживать фильтрацию по продукту с выбором
        многих продуктов (checkbox-выбор)
  \item Kanban-доска должна поддерживать фильтрацию по команде с выбором
        многих команд (checkbox-выбор)
  \item Kanban-доска должна предоставлять возможность изменять статус
        команд с помощью drag-n-drop между колонками статусов
\end{enumerate}

\subsection{Требования к редактору текста}
\begin{enumerate}[label=\textbf{AER\arabic*}.]
  \item Редактор текста должен предоставлять возможность выделять текст
        \textbf{жирным} шрифтом, \textit{курсивом} и \texttt{моноширинным} шрифтом
  \item Редактор текста должен поддерживать вставку ссылок на веб-сайты
  \item Редактор текста должен поддерживать отображение заголовков 6-ти разных
        уровней
\end{enumerate}

\subsection{Требования к Workflow}
\begin{enumerate}[label=\textbf{WFR\arabic*}.]
  \item Система должна предоставлять следующие группы статусов:
        \begin{enumerate*}
          \item <<Backlog>>: Backlog - задача не находится ни в одном из бэклогов спринта.
                To Do - задча находится в бэклоге спринта \item <<Started>>: In Progress -
                задача взята в работу, In Review - задача находится в ревью, Ready To Merge -
                задача прошла ревью и готова к мержу \item <<Completed>>: Done - задача
                выполнена и замержена \item <<Canceled>>: Canceled - задача отменена, Couldn't
                reproduce - не удалось воспроизвести задачу, Won't Fix - задача не будет
                исправлена, Duplicate - задача является дубликатом \end{enumerate*} \item Я,
        как Scrum Master, хочу выбирать какие из вышеперечисленных статусов могут быть
        в дальнейшем назначены задачам \item Я, как Scrum Master, хочу добавлять
        статусы в каждую из вышеперечисленных групп.
  \item Система должна ограничивать максимальное количество активных статусов до 7-ми
\end{enumerate}


\section{Нефункциональные требования}
\begin{enumerate}[label=\textbf{NFR\arabic*}.]
  \item Система должна поддерживать отображения без нарушения работы функциональности
        и дизайна в браузерах Chrome 130+, Mozilla 120+
\end{enumerate}


