\usepackage[T2A]{fontenc}
\usepackage[utf8]{inputenc}
\usepackage{hyperref}
\usepackage{graphicx}
\usepackage{minted}
\usepackage[english,russian]{babel}
\usepackage{fancyhdr}
\usepackage{caption}

%
\usepackage{tabularx}
\usepackage{array}
\usepackage{longtable}
\usepackage{tabularray}
\usepackage{tikz}

% Inline lists with enumerate*
\usepackage[inline]{enumitem}
\usepackage{calc}% используется для сложения длин
\setlist[itemize]{%
  leftmargin=0pt, % согласно ГОСТ, вторая строка элемента списка должна
  % начинаться без абзацного отступа.
  %
  % Отступ первой строки от левого края будет равен: абз. отступ + ' -- '.
  % Хочу заметить, что это результат не совсем точный, но выглядит неплохо.
  % (другого способа настроить нужный отступ не нашел. ¯\_(ツ)_/¯). TODO.
  itemindent={\the\parindent + \widthof{\ --\ }},
  itemsep=0cm, % лол, не знаю что это;
  nolistsep, % убираем большие скачки по вертикали;
  label=--% используем короткое тире вместо bullets.
}
\setlist[enumerate]{%
  % Итоговый отступ элемента от левого края будет: 1.5cm + ширина ' 1) '.
  % В результате на элементе из двух цифр, типа '10)', может вылезти за края 😨.
  leftmargin=0pt,
  itemindent={\the\parindent + \widthof{\ 1)\ }},
  itemsep=0cm,
  nolistsep,
  label={\arabic*)}%
}

% For... glossary
\usepackage[toc,translate=babel]{glossaries}

% To set custom label for enumerate
\usepackage{enumitem}

\usepackage{geometry}
\geometry{right=15mm}
\geometry{left=30mm}
\geometry{bottom=20mm}
\geometry{top=20mm}
\geometry{ignorefoot}% считать от нижней границы текста

\pagestyle{fancy}
\fancyhead{}
\fancyfoot[C]{\thepage}

\renewcommand{\headrulewidth}{0pt}
\fancypagestyle{firststyle}
{
  \fancyhead{}
  \fancyfoot[C]{г. Санкт-Петербург\\2024 г.}
}

\author{Егор Федоров \and Андрей Карабанов}
\title{Software Requirements for Zadachnik}

\usepackage{setspace}
\linespread{1.5}

\setlength{\parindent}{1.25cm} % отступ для абзаца
\usepackage{indentfirst}
\usepackage[inkscapeformat=png]{svg}

