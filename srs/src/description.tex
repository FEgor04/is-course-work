\section{Описание предметной области}
В системе Scrum люди делятся по ролям
Developer, Scrum Master, Product Owner.
Для организации работы создаются задачи.
Задачи могут быть трех типов: \glsdisp{epic}{эпик}, \glsdisp{story}{стори},
\glsdisp{task}{задача}.
Задачи имеют \glsdisp{task-attribute}{атрибуты}.
Задачи также могут иметь сложные связи между собой, например
задача может блокировать другую или задача может являться клоном другой,
задача может быть подзадачей стори или эпика.
Стори может быть подзадачей эпика.

Также в Scrum имеются 4 церемонии-встречи: daily meeting, sprint
planning, spring review, sprint retro.
Встречи также могут иметь минутки -- записи о всем, что произошло на встрече.
Для оценки времени на выполнение задач в Scrum используются \glsdisp{storypoint}{сторипоинты}.

Люди объединяются в команды. Команда обязательно имеет одну и только одну доску,
в которой отображаются задачи всех продуктов, в разработке
которых принимает участие команда.
Команда имеет особый идентификатор.
Список задач команды в основном управляется scrum master.

При этом команда может участвовать в нескольких продуктах.
Задачи создаются в продуктах и тем самым формируют его бэклог.
Список задач продукта в основном управляется product owner.

Для выполнения задач команды создают спринты и на каждый
спринт формируют бэклог спринта.
В процессе спринта команда \emph{может} создать релиз.
По окончанию спринта команда \emph{обязана} создать релиз.

Релиз -- описание всех изменений в продукте.
Описание изменений состоит из списка всех задач,
закрытых при релизе и типов этих изменений (добавлено, изменено и т.п., см. сайт \url{keepachangelog.org}).
Релиз также может содержать release notes - дополнительное описание релиза, например
краткое описание всех изменений.

\section{Описание информационной системы}
Информационная система Zadachnik -- система отслеживания багов и
управления проектами.
Основное предназначение системы -- организовывать управление задачами и багами
в IT проектах, реализующих Scrum.

Информационная система позволит решить следующие задачи:
\begin{itemize}
  \item Отслеживание задач в проекте
  \item Отслеживание связей между задачами
  \item Распределение задач между участникам
  \item Оценка времени на выполнение задач
  \item Оценка времени на выполнение всего проекта
  \item Сбор аналитики по времени выполнения задач
  \item Фасилитация и организация sprint planning, daily scrum, spring review и spring retro.
\end{itemize}

\subsection{Классы и характеристики пользователей}
Так как система построена для использования в Scrum-командах, то в ней
можно выделить следующие классы пользователей. Эти классы в основном выделены из
классических ролей в Scrum, в качестве источника был использован веб-сайт scrumtrek.ru
\begin{itemize}
  \item Product Owner. Отвечает за максимизацию ценности продукта, получаемого в результате работы Scrum-команды.
        В его обязанности также входит курирование и приоритизация бэклога продукта.
  \item Scrum Master. Является лидером-слугой (Servant Leader) для Скрам-команды и для организации в целом.
        Обучает команду устранять препятствия, является коучем команды и фасилитирует Мероприятия Скрама.
        Фактически является владельцем процесса, ответственным за эффективную работу команды.
  \item Developer. Разработчики -- это люди, работающие над Элементами Бэклога Спринта.
        Они имеют все необходимые компетенции, чтобы каждый Спринт создавать работающий Инкремент Продукта.
\end{itemize}


